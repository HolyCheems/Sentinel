\section*{Popper.\+js}

{\bfseries{A library used to position poppers in web applications.}} 

\href{https://travis-ci.org/FezVrasta/popper.js/branches}{\texttt{ }}  \href{https://www.bithound.io/github/FezVrasta/popper.js}{\texttt{ }} \href{https://codeclimate.com/github/FezVrasta/popper.js/coverage}{\texttt{ }} \href{https://gitter.im/FezVrasta/popper.js}{\texttt{ }} ~\newline
 \href{https://saucelabs.com/u/popperjs}{\texttt{ }} 



\subsection*{Wut? Poppers?}

A popper is an element on the screen which \char`\"{}pops out\char`\"{} from the natural flow of your application.

Common examples of poppers are tooltips, popovers and drop-\/downs.

\subsection*{So, yet another tooltip library?}

Well, basically, {\bfseries{no}}.

Popper.\+js is a {\bfseries{positioning engine}}, its purpose is to calculate the position of an element to make it possible to position it near a given reference element.

The engine is completely modular and most of its features are implemented as {\bfseries{modifiers}} (similar to middlewares or plugins).

The whole code base is written in E\+S2015 and its features are automatically tested on real browsers thanks to \href{https://saucelabs.com/}{\texttt{ Sauce\+Labs}} and \href{https://travis-ci.org/}{\texttt{ Travis\+CI}}.

Popper.\+js has zero dependencies. No j\+Query, no Lo\+Dash, nothing.

It\textquotesingle{}s used by big companies like \href{https://getbootstrap.com/}{\texttt{ Twitter in Bootstrap v4}}, \href{https://github.com/OneNoteDev/WebClipper}{\texttt{ Microsoft in Web\+Clipper}} and \href{https://aui-cdn.atlassian.com/atlaskit/registry/}{\texttt{ Atlassian in Atlas\+Kit}}.

\subsubsection*{Popper.\+js}

This is the engine, the library that computes and, optionally, applies the styles to the poppers.

Some of the key points are\+:


\begin{DoxyItemize}
\item Position elements keeping them in their original D\+OM context (doesn\textquotesingle{}t mess with your D\+O\+M!);
\item Allows to export the computed informations to integrate with React and other view libraries;
\item Supports Shadow D\+OM elements;
\item Completely customizable thanks to the modifiers based structure;
\end{DoxyItemize}

Visit our \href{https://fezvrasta.github.io/popper.js}{\texttt{ project page}} to see a lot of examples of what you can do with Popper.\+js!

Find the documentation here.

\subsubsection*{Tooltip.\+js}

Since lots of users just need a simple way to integrate powerful tooltips in their projects, we created {\bfseries{Tooltip.\+js}}.

It\textquotesingle{}s a small library that makes it easy to automatically create tooltips using as engine Popper.\+js.

Its A\+PI is almost identical to the famous tooltip system of Bootstrap, in this way it will be easy to integrate it in your projects.

The tooltips generated by Tooltip.\+js are accessible thanks to the {\ttfamily aria} tags.

Find the documentation here.

\subsection*{Installation}

Popper.\+js is available on the following package managers and C\+D\+Ns\+:

\tabulinesep=1mm
\begin{longtabu}spread 0pt [c]{*{2}{|X[-1]}|}
\hline
\cellcolor{\tableheadbgcolor}\textbf{ Source  }&\cellcolor{\tableheadbgcolor}\textbf{ }\\\cline{1-2}
\endfirsthead
\hline
\endfoot
\hline
\cellcolor{\tableheadbgcolor}\textbf{ Source  }&\cellcolor{\tableheadbgcolor}\textbf{ }\\\cline{1-2}
\endhead
npm  &{\ttfamily npm install popper.\+js -\/-\/save}   \\\cline{1-2}
yarn  &{\ttfamily yarn add popper.\+js}   \\\cline{1-2}
Nu\+Get  &{\ttfamily PM$>$ Install-\/\+Package popper.\+js}   \\\cline{1-2}
Bower  &{\ttfamily bower install popper.\+js -\/-\/save}   \\\cline{1-2}
unpkg  &\href{https://unpkg.com/popper.js}{\texttt{ {\ttfamily https\+://unpkg.\+com/popper.\+js}}}   \\\cline{1-2}
cdnjs  &\href{https://cdnjs.com/libraries/popper.js}{\texttt{ {\ttfamily https\+://cdnjs.\+com/libraries/popper.\+js}}}   \\\cline{1-2}
\end{longtabu}


Tooltip.\+js as well\+:

\tabulinesep=1mm
\begin{longtabu}spread 0pt [c]{*{2}{|X[-1]}|}
\hline
\cellcolor{\tableheadbgcolor}\textbf{ Source  }&\cellcolor{\tableheadbgcolor}\textbf{ }\\\cline{1-2}
\endfirsthead
\hline
\endfoot
\hline
\cellcolor{\tableheadbgcolor}\textbf{ Source  }&\cellcolor{\tableheadbgcolor}\textbf{ }\\\cline{1-2}
\endhead
npm  &{\ttfamily npm install tooltip.\+js -\/-\/save}   \\\cline{1-2}
yarn  &{\ttfamily yarn add tooltip.\+js}   \\\cline{1-2}
Bower$\ast$  &{\ttfamily bower install tooltip.\+js=\href{https://unpkg.com/tooltip.js}{\texttt{ https\+://unpkg.\+com/tooltip.\+js}} -\/-\/save}   \\\cline{1-2}
unpkg  &\href{https://unpkg.com/tooltip.js}{\texttt{ {\ttfamily https\+://unpkg.\+com/tooltip.\+js}}}   \\\cline{1-2}
cdnjs  &\href{https://cdnjs.com/libraries/popper.js}{\texttt{ {\ttfamily https\+://cdnjs.\+com/libraries/popper.\+js}}}   \\\cline{1-2}
\end{longtabu}


$\ast$\+: Bower isn\textquotesingle{}t officially supported, it can be used to install Tooltip.\+js only trough the unpkg.\+com C\+DN. This method has the limitation of not being able to define a specific version of the library. Bower and Popper.\+js suggests to use npm or Yarn for your projects.

For more info, \href{https://github.com/FezVrasta/popper.js/issues/390}{\texttt{ read the related issue}}.

\subsubsection*{Dist targets}

Popper.\+js is currently shipped with 3 targets in mind\+: U\+MD, E\+SM and E\+S\+Next.


\begin{DoxyItemize}
\item U\+MD -\/ Universal Module Definition\+: A\+MD, Require\+JS and globals;
\item E\+SM -\/ ES Modules\+: For webpack/\+Rollup or browser supporting the spec;
\item E\+S\+Next\+: Available in {\ttfamily dist/}, can be used with webpack and {\ttfamily babel-\/preset-\/env};
\end{DoxyItemize}

Make sure to use the right one for your needs. If you want to import it with a {\ttfamily $<$script$>$} tag, use U\+MD.

\subsection*{Usage}

Given an existing popper D\+OM node, ask Popper.\+js to position it near its button


\begin{DoxyCode}{0}
\DoxyCodeLine{var reference = document.querySelector('.my-button');}
\DoxyCodeLine{var popper = document.querySelector('.my-popper');}
\DoxyCodeLine{var anotherPopper = new Popper(}
\DoxyCodeLine{    reference,}
\DoxyCodeLine{    popper,}
\DoxyCodeLine{    \{}
\DoxyCodeLine{        // popper options here}
\DoxyCodeLine{    \}}
\DoxyCodeLine{);}
\end{DoxyCode}


\subsubsection*{Callbacks}

Popper.\+js supports two kinds of callbacks, the {\ttfamily on\+Create} callback is called after the popper has been initalized. The {\ttfamily on\+Update} one is called on any subsequent update.


\begin{DoxyCode}{0}
\DoxyCodeLine{const reference = document.querySelector('.my-button');}
\DoxyCodeLine{const popper = document.querySelector('.my-popper');}
\DoxyCodeLine{new Popper(reference, popper, \{}
\DoxyCodeLine{    onCreate: (data) => \{}
\DoxyCodeLine{        // data is an object containing all the informations computed}
\DoxyCodeLine{        // by Popper.js and used to style the popper and its arrow}
\DoxyCodeLine{        // The complete description is available in Popper.js documentation}
\DoxyCodeLine{    \},}
\DoxyCodeLine{    onUpdate: (data) => \{}
\DoxyCodeLine{        // same as `onCreate` but called on subsequent updates}
\DoxyCodeLine{    \}}
\DoxyCodeLine{\});}
\end{DoxyCode}


\subsubsection*{Writing your own modifiers}

Popper.\+js is based on a \char`\"{}plugin-\/like\char`\"{} architecture, most of its features are fully encapsulated \char`\"{}modifiers\char`\"{}.

A modifier is a function that is called each time Popper.\+js needs to compute the position of the popper. For this reason, modifiers should be very performant to avoid bottlenecks.

To learn how to create a modifier, \href{docs/_includes/popper-documentation.md\#modifiers--object}{\texttt{ read the modifiers documentation}}

\subsubsection*{React, Vue.\+js, Angular, Angular\+JS, Ember.\+js (etc...) integration}

Integrating 3rd party libraries in React or other libraries can be a pain because they usually alter the D\+OM and drive the libraries crazy.

Popper.\+js limits all its D\+OM modifications inside the {\ttfamily apply\+Style} modifier, you can simply disable it and manually apply the popper coordinates using your library of choice.

For a comprehensive list of libraries that let you use Popper.\+js into existing frameworks, visit the M\+E\+N\+T\+I\+O\+NS page.

Alternatively, you may even override your own {\ttfamily apply\+Styles} with your custom one and integrate Popper.\+js by yourself!


\begin{DoxyCode}{0}
\DoxyCodeLine{function applyReactStyle(data) \{}
\DoxyCodeLine{    // export data in your framework and use its content to apply the style to your popper}
\DoxyCodeLine{\};}
\DoxyCodeLine{}
\DoxyCodeLine{const reference = document.querySelector('.my-button');}
\DoxyCodeLine{const popper = document.querySelector('.my-popper');}
\DoxyCodeLine{new Popper(reference, popper, \{}
\DoxyCodeLine{    modifiers: \{}
\DoxyCodeLine{        applyStyle: \{ enabled: false \},}
\DoxyCodeLine{        applyReactStyle: \{}
\DoxyCodeLine{            enabled: true,}
\DoxyCodeLine{            fn: applyReactStyle,}
\DoxyCodeLine{            order: 800,}
\DoxyCodeLine{        \},}
\DoxyCodeLine{    \},}
\DoxyCodeLine{\});}
\end{DoxyCode}


\subsubsection*{Migration from Popper.\+js v0}

Since the A\+PI changed, we prepared some migration instructions to make it easy to upgrade to Popper.\+js v1.

\href{https://github.com/FezVrasta/popper.js/issues/62}{\texttt{ https\+://github.\+com/\+Fez\+Vrasta/popper.\+js/issues/62}}

Feel free to comment inside the issue if you have any questions.

\subsubsection*{Performances}

Popper.\+js is very performant. It usually takes 0.\+5ms to compute a popper\textquotesingle{}s position (on an i\+Mac with 3.\+5G G\+Hz Intel Core i5).

This means that it will not cause any \href{https://www.chromium.org/developers/how-tos/trace-event-profiling-tool/anatomy-of-jank}{\texttt{ jank}}, leading to a smooth user experience.

\subsection*{Notes}

\subsubsection*{Libraries using Popper.\+js}

The aim of Popper.\+js is to provide a stable and powerful positioning engine ready to be used in 3rd party libraries.

Visit the M\+E\+N\+T\+I\+O\+NS page for an updated list of projects.

\subsubsection*{Credits}

I want to thank some friends and projects for the work they did\+:


\begin{DoxyItemize}
\item \href{https://github.com/AndreaScn}{\texttt{ @\+Andrea\+Scn}} for his work on the Git\+Hub Page and the manual testing he did during the development;
\item \href{https://github.com/vampolo}{\texttt{ @vampolo}} for the original idea and for the name of the library;
\item \href{https://github.com/Draios}{\texttt{ Sysdig}} for all the awesome things I learned during these years that made it possible for me to write this library;
\item \href{http://github.hubspot.com/tether/}{\texttt{ Tether.\+js}} for having inspired me in writing a positioning library ready for the real world;
\item \href{https://github.com/FezVrasta/popper.js/graphs/contributors}{\texttt{ The Contributors}} for their much appreciated Pull Requests and bug reports;
\item {\bfseries{you}} for the star you\textquotesingle{}ll give to this project and for being so awesome to give this project a try 🙂
\end{DoxyItemize}

\subsubsection*{Copyright and license}

Code and documentation copyright 2016 {\bfseries{Federico Zivolo}}. Code released under the M\+IT license. Docs released under Creative Commons. 